\documentclass{article}
\usepackage{graphicx}
\usepackage{float}
\usepackage[top=1in, bottom=1in, left=0.8in, right=0.8in]{geometry}

\title{Explicit simulation of diffusion into macropores in spherical beads in fixed-bed column}
\author{Guoping Tang}
%\date{September 1994}
\begin{document}
   \maketitle

\section{Problem statement}
The advection, diffusion, and sorption of solute in a fixed-bed column can be described by 
\begin{equation}
\frac{\partial (1 + k_p) c}{\partial t} = - vc + (\tau D_m + \lambda v) \frac{\partial^2 c}{\partial x^2} + \sum \frac{\tau D_m}{r^2}\frac{\partial }{\partial
r}\left(r^2\frac{\partial c}{\partial r}\right), 
\label{eq:ade}
\end{equation}
with initial concentration 
\begin{equation}
c(x, r, t=0) = 0,
\end{equation}
and boundary conditions
\begin{equation}
c(x=0, r = 0,t) = c_0,
\end{equation}
and
\begin{equation}
\frac{\partial c}{\partial x} (x=L, r=0,t) = 0.
\end{equation}
Here $c$ is the aqueous concentration; $k_p$ is the partition coefficient (dissolved vs. sorbed concentration, 0 in the mobile fraction); $v$ is the advection velocity; $D_m$ is the free water molecular diffusion coefficient; $\tau$ is the tortuosity factor; $\lambda$ is the dispersivity; $x$ is the distance from the inlet; $r$ is the distance from the outer surface into the sphere; $L$ is the length of the column, and $\sum$ denotes the summation of the beads in a unit volume. 

\section{Numerical solution}
\subsection{Finite difference formulae}
Discretizing the column in $x$ direction into $n_x$ uniform cells and the sphere into $n_r$  cells, and adding a film cell around the sphere, the numerical grid cells are numbered from the $n_x$ column cells first, starting from inlet to the outlet, and then from the outer to the center of the sphere cells. The total number of grid cells is $n_x(n_r + 2)$. Suppose that a column is filled with $V_b$ spherical resin beads of uniform radius $R_b$, the number of beads in a column grid cell is
\begin{equation}
n_b = \frac{3V_b}{4\pi R_b^3n_x} .
\end{equation}
Using upwind for advection and center difference for diffusion, mass balance equation for the first cell is
\begin{equation}
\frac{\partial c_1}{\partial t}\theta_m A \Delta x = v\theta_m A(c_0 - c_1) + \frac{(\tau D_m+\lambda v )\theta_m A}{\Delta x} \left(c_0 -2 c_1 + c_2\right) + n_b D_m \frac{A_0}{0.5h} (c_{n_x+1} - c_1).
\end{equation}
Here $\theta_m$ is the mobile water content; $A=\pi R_c^2$ is the column section area; $\theta_b$  is the volumetric water content of the macroporous resin; $A_0$ is the surface area of the bead plus film, and $h$ is the film thickness. For a column with radius $R_c$ that is filled with $V_b$ resin, $\theta_m = 1 - V_b/V_c$, with $V_c = AL$. Dividing both sides by $\theta_mA\Delta x = \theta_m \Delta V$, 
\begin{equation}
\frac{\partial c_1}{\partial t}= \frac{v}{\Delta x}(c_0 - c_1) + \frac{(\tau D_m+\lambda v )}{\Delta x^2} \left(c_0 -2 c_1 + c_2\right) + \frac{n_b D_m A_0}{\Delta V \theta_m 0.5h} (c_{n_x+1} - c_1).
\end{equation}
Denote 
\begin{equation}
\alpha = \frac{v \Delta t}{\Delta x},
\end{equation}
\begin{equation}
\beta = \frac{(\tau D_m + \lambda v) \Delta t}{\Delta x^2}, 
\end{equation}
and
\begin{equation}
\gamma = \frac{n_b}{\Delta V \theta_m}\frac{D_mA_0\Delta t}{0.5h} =  \frac{n_b}{\Delta V \theta_m}\eta_1,
\end{equation}
and use the backward Euler method for time discretization with $\Delta t$ as the time step size from time step $k$ to $k+1$, it reduces to
\begin{equation}
(1 + \alpha + 2 \beta + \gamma) c_1^{k+1}  - \beta c_2^{k+1} - \gamma c_{n_x+1}^{k+1} = c_1^k + (\alpha + \beta) c_0.
\end{equation}
For the second cell in the column, 
\begin{equation}
-(\alpha + \beta)c_1^{k+1}  + (1 +  \alpha + 2 \beta + \gamma ) c_2^{k+1}  - \beta c_3^{k+1}- \gamma c_{n_x+2}^{k+1} =  c_2^k .
\end{equation}
For the last cell in the column,
\begin{equation}
-(\alpha + \beta)c_{n_x-1}^{k+1}  + (1 +  \alpha + \beta + \gamma ) c_{n_x}^{k+1}  - \gamma c_{n_x+n_r}^{k+1} =  c_{n_x}^k,
\end{equation}
as the zero concentration gradient outlet boundary condition is implemented with $c_{n_x} = c_{n_x+1}$ by adding a hypothetical extra grid cell \cite{Zheng2002}.

For the stagnant film water cell surrounding the sphere, 
\begin{equation}
\frac{\partial c_{n_x+1}}{\partial t}\Delta V_0 =  A_1 D_m \frac{c_1 - c_{n_x+1}}{0.5h} +  \theta_b A_1\tau D_m \frac{c_{2n_x+1} - c_{n_x+1}}{0.5(h + \Delta r_1)}.
\end{equation}

For the outer grid cell in the sphere, the mass conservation equation is
\begin{equation}
\frac{\partial c_{n_x+2}}{\partial t}\theta_b\Delta V_1 (1 + k_p) =  \theta_b A_1\tau D_m \frac{c_{n_x+1} - c_{2n_x+1}}{0.5(h+\Delta r_1)} +  \theta_b A_2\tau D_m \frac{c_{3n_x+1} - c_{2n_x+1}}{0.5(\Delta r_1 + \Delta r_2)}.
\end{equation}

Denote 
\begin{equation}
\omega_0 = \frac{1}{\Delta V_0},
\end{equation}
\begin{equation}
\eta_0 = \frac{D_mA_0\Delta t}{\Delta R_0}, 
\end{equation}
with $\Delta R_0 = h/2$, and 
\begin{equation}
\omega_i = \frac{1}{\Delta V_i (1+k_p)},
\end{equation}
\begin{equation}
\eta_i = \frac{\tau D_mA_i\Delta t}{\Delta R_i}, 
\end{equation}
with $\Delta R_1 = 0.5(h + \Delta r_1)$ for the outer grid cell, $\Delta R_i = 0.5 (\Delta r_i + \Delta r_{i+1})$ for the intermediate grid cells, and $\Delta R_i = 0.5 \Delta r_{n_r}$ for the grid cell at the center of the sphere. 
Use the same backward Euler method for time discretization, it simplifies to 
\begin{equation}
-\omega_0 \eta_0 c_1^{k+1}  + (1 + \omega_0\eta_0 + \omega_0\eta_1) c_{n_x+1}^{k+1} -  \omega_1 \eta_1 c_{2n_x+1}^{k+1} =   c_{n_x+1}^{k},
\end{equation}
for the film cell, and for the outer grid cell,
\begin{equation}
-\omega_1 \eta_1 c_1^{k+2}  + (1 + \omega_1\eta_1 + \omega_1\eta_2) c_{n_x+2}^{k+1} -  \omega_1 \eta_2 c_{2n_x+2}^{k+1} =   c_{n_x+1}^{k}.
\end{equation}
For the second grid cell in the sphere,
\begin{equation}
-\omega_2 \eta_2 c_{n_x+3}^{k} + (1 + \omega_2 \eta_2 + \omega_2 \eta_3)  c_{2n_x+3}^{k+1} -  \omega_2 \eta_3 c_{3n_x+3}^{k+1}=  c_{2n_x+3}^{k}.
\end{equation}
For the grid cell in the center of the sphere,
\begin{equation}
-\omega_{n_r} \eta_{n_r} c_{n_x(n_r-1)+2}^{k} +  (1 + \omega_{n_r} \eta_{n_r}) c_{n_xn_r+2}^{k+1} =  c_{n_xn_r+2}^{k}.
\end{equation}

\subsection{Matrix equation}
Assemble these equations into a matrix equation,
\begin{equation}
\mathbf{A}\mathbf{x}=\mathbf{b}
\end{equation}

For the first row, $\mathbf{A}(1, 1) = 1 + \alpha + 2 \beta + \gamma$, $\mathbf{A}(1, 2) = -\beta$,  $\mathbf{A}(1, n_x+1) = -\gamma$, and $\mathbf{b}(1) = c_1^k + (\alpha+\beta)c_0$.   
From the second row to the second last row in the column ($i=2$ to $n_x-1$),  $\mathbf{A}(i, i-1) = -\alpha - \beta$, $\mathbf{A}(i, i) = 1 + \alpha + 2 \beta + \gamma$, $\mathbf{A}(i, i+1) = - \beta$,  $\mathbf{A}(i, n_x+i) = -\gamma$, and $\mathbf{b}(i) = c_i^k$. 
For $i = n_x$ starting from the outer sphere grid cell, $\mathbf{A}(i, i-1) = -\alpha - \beta$, $\mathbf{A}(i, i) = 1 + \alpha + \beta + \gamma$,  $\mathbf{A}(i, n_r+i) = -\gamma$, and $\mathbf{b}(i) = c_i^k$.
For $i >n_x$ and $\leq n_x\times n_r$, $\mathbf{A}(i, i-n_x) = -\omega_j \eta_j $, $\mathbf{A}(i, i) = 1 + \omega_j \eta_j + \omega_j \eta_{j+1} $,  $\mathbf{A}(i, i + n_r) = -\omega_j \eta_{j+1}$, and $\mathbf{b}(i) = c_i^k$. Where $j = i\%n_r$ (remainder if $i/n_r$ in c).
For $i >n_xn_r$ , $\mathbf{A}(i, i-n_x) = -\omega_j \eta_j $, $\mathbf{A}(i, i) = 1 + \omega_j \eta_j $,  and $\mathbf{b}(i) = c_i^k$. 
This is summarized in Table \ref{tab:assembly}.
\noindent
\begin{table}[H]
\label{tab:assembly}
\caption{Matrix assembly ($k=j\%n_x$)}
\begin{tabular}{ c | cccc | cccc }
\hline
Row & \multicolumn{4}{c|}{Entries} & \multicolumn{4}{c}{Column} \\
\hline
1 & $1+\alpha +2\beta+\gamma$      & $-\beta$ & $-\gamma$ & & 1 & 2 & $n_x+1$ & \\
i  & $-\alpha-\beta$ & $1+\alpha +2\beta+\gamma$      & $-\beta$ & $-\gamma$ & i-1 & i & i+1 & $n_x+i$ \\
$n_x$ & $-\alpha-\beta$ & $1+\alpha +\beta+\gamma$      & $-\gamma$ & & $n_x-1$ & $n_x$ & $n_x+n_r$ \\
\hline
$n_x+1$ & $-\omega_1\eta_1$ & $1+\omega_1\eta_1+\omega_1\eta_2$  & $-\omega_1\eta_2$    &   & 1 & $n_x+1$ & $2n_x+1$ &  \\
$j$ & $-\omega_k\eta_k$ & $1+\omega_k(\eta_k+\eta_{k+1})$  & $-\omega_k\eta_{k+1}$    &   & $j-n_x$ & j & $n_x+j$ &  \\
$n_xn_r$ & $-\omega_{n_r}\eta_{n_r}$ & $1+\omega_{n_r}(\eta_{n_r}+\eta_{n_r+1})$  & $-\omega_{n_r}\eta_{n_r+1}$    &   & $n_x(n_r-1)$ & $n_xn_r$ & $n_x(n_r+1)$ &  \\
\hline
$n_xn_r+1$ & $-\omega_{n_r}\eta_{n_r}$ & $1+\omega_{n_r}\eta_{n_r}$  &  &   & $n_x(n_r-1)+1$ & $n_xn_r+1$ & &  \\
$n_xn_r+l$ & $-\omega_{n_r}\eta_{n_r}$ & $1+\omega_{n_r}\eta_{n_r}$  &  &   & $n_x(n_r-1)+l$ & $n_xn_r+l$ & &  \\
$n_x(n_r+1)$ & $-\omega_{n_r}\eta_{n_r}$ & $1+\omega_{n_r}\eta_{n_r}$  &  &   & $n_xn_r+1$ & $n_x(n_r+1)$ &  &  \\
\hline
\end{tabular}
\end{table}
\subsection{Discretization of the sphere}
The sphere can be discretized with uniform distance or volume. For the latter, the radius values follow
\begin{equation}
\frac{4\pi r_i^3}{3} - \frac{4\pi (r_i-\Delta r_i)^3}{3} = \frac{4\pi R_b^3}{3n_r}.
\end{equation}
The $r_i$ and $\Delta r_i$ are calculated by
\begin{equation}
\Delta r_i = r_i - (r_i^3 - R_b^3/n_r)^{1/3},
\end{equation}
starting from $r_1 = R_b$.

\subsection{Accuracy}
The backward Euler method makes the solution unconditionally stable.
For the upwind for advection, center difference for diffusion, and backward difference for time, the numerical diffusion coefficient
\begin{equation}
\frac{D_n}{D_0} = 0.5 P_e (1 + \alpha),
\end{equation}
with the Peclet number $P_e=\frac{v \Delta t}{D_e+v\lambda}$, and $D_0$ as the free water molecular diffusion coefficient. A Courant number less than 1, and a small Peclet number is needed for accuracy. The numerical diffusion coefficient can be subtracted from the physical diffusion coefficient to increase accuracy.

\section{Tests}
\subsection{Advection dispersion equation}
For first-type Dirichlet inlet boudnary condition, the analytical solution for the one dimensional convection diffusion equation is is \cite{Lapidus1952,Ogata1961}
\begin{equation}
\frac{c-c_i}{c_0-c_i} = \frac{1}{2}\left[\mathrm{erfc}\left(\frac{x-vt}{2\sqrt{Dt}}\right) + \exp\left(\frac{xv}{D}\right)\mathrm{erfc}\left(\frac{x+vt}{2\sqrt{Dt}}\right)\right].
\end{equation}
Where $c_i$ and $c_0$ are the initial and inlet concentration.

For length = 5, diffusion coefficient = 0.01, dispersivity = 0, we use $D_f/D_0$ = 0 to prevent diffusion into sphere, the numerical solution is compared with analytical solution in Figs. \ref{fig:test1prof} and \ref{fig:test1comp}. As the grid Peclet number decreases to below 1, the numerical solution gets close to the analytical solution. Coarse discretization results in additional numerical diffusion.

\begin{figure}[htb]
\centerline{\includegraphics[width=1.0\textwidth]{../tests/test1/comp.pdf}}
\caption{Verification with analytical solution for advection diffusion equation: breakthrough curves. $Pe = v\Delta t/D$ is the grid Peclet number. The Courant number $Cr = v\Delta t\Delta x$ = 1.}
\label{fig:test1comp}
\end{figure}

\begin{figure}[htb]
\centerline{\includegraphics[width=1.0\textwidth]{../tests/test1/prof.pdf}}
\caption{Verification with analytical solution for advection diffusion equation: profile history. $Pe = v\Delta t/D$ is the grid Peclet number. The Courant number $Cr = v\Delta t\Delta x$ = 1.}
\label{fig:test1prof}
\end{figure}

\subsection{Diffusion into a sphere with constant concentration at the surface}
The analytical solution for diffusion into a sphere with constant surface concentration is \cite{Crank1975} 
\begin{equation}
\frac{c-c_i}{c_0-c_i} = 1 + \frac{2a}{\pi r} \sum_{n=1}^\infty \frac{(-1)^n}{n} \sin\frac{n\pi r}{a}\exp\left(\frac{Dn^2\pi^2t}{a^2}\right).
\end{equation}
where $c_i$ and $c_0$ are the initial and surface boundary concentration, $r$ is the distance from the center of the sphere, and $a$ is the radius of the sphere. For r = 0, 
\begin{equation}
\frac{c-c_i}{c_0-c_i} = 1 + 2 \sum_{n=1}^\infty (-1)^n \exp\left(\frac{Dn^2\pi^2t}{a^2}\right).
\end{equation}

For $a = 0.028$ cm, $D_e = 2\times 10^{-6}$ cm$^2$/s, we set the intial and inlet concentration at 1, and specify a fast flow rate 50 cm/s to keep the concentration at the surface of sphere at 1. The numerical solutions with discretizing the sphere with 10, 20, and 100 uniform radial distance are compared with the analytical solution in Fig. \ref{fig:test2prof}. The numerical solution gets close to the analytical solution when the discretization is fine. Coarse discretization results in additional numerical diffusion. 

\begin{figure}[htb]
\centerline{\includegraphics[width=1.0\textwidth]{../tests/test2/prof.pdf}}
\caption{Verification with analytical solution for diffusion into a sphere: profile history.}
\label{fig:test2prof}
\end{figure}


\bibliography{cxrt}{}
\bibliographystyle{plain}

\end{document}


